Assignments 1-8 have been completed. This can be verified through SkyHiGh/OpenStack.

Imagine a three-team architecture with 3 maintenance groups: Application-maintenance, database-maintenance and storage/infrastructure maintenance. Choose four tasks or events which would require coordination between all three groups, even if the task is only performed by one of them. Describe all four with a few lines, resulting in 0.5-1 page with text.

\subsection{8. Scenario \#1: Database relocation}
During a database relocation, all three teams have to coordinate with one another to work out where the database is moving to, DNS reconfiguration and resolution, and, if moving to a new software, back-end or even front-end code may need to be reconfigured to get the requested information in and out of the database again. This has to happen seamlessly, as to not cause any more disturbance than is absolutely necessary for the users, requiring all teams to plan in advance and coordinate their actions and timing with one another.

\subsection{8. Scenario \#2: Arson}
In case the datacenter is set on fire and the fire reaches such critical mass causing irrepairable damage to hardware, an incredible amount of information may be lost - unless, of course, the company has been consistent and frequent in their off-site backups. In case they haven't, all three teams must work together and recreate the most recent backup in a working configuration, knowing that every minute is costing the company massive revenue - this puts the entire team under a lot of pressure, and there is few other scenarios in which coordination and teamwork is as important as this. Even with consistent and frequent off-shore backup, such a sudden event will require quick response, trained personel, and a tried and tested contingency plan for natural disasters.

\subsection{8. Scenario \#3: APT}
In a dark cellar, with the floor covered in empty bags of Dorito's, Domino's pizza boxes and empty cans of Mountain Dew, somewhere in the Alaskan wilderness, a group of 37 twenty- and thirty-somethings had spent the past two weeks. Their objective was straight-forward, it's implementation not so much. After days of passive reconnaissance, they had been able to map large parts of the inner infrastructure of the Bookface social propaganda site, and gathered information on most of the employees; this was followed up by almost a week of active, but irregular, reconnaissance in an attempt to map the entire infrastructure of the Bookface datacenters. 

The windows were covered, and the walls were painted black - not so much to hide their presence, but to allow each and everyone in the large underground hall the creative freedom of being able to draw on the walls. The dark color of the walls was merely a result of the chalkboard paint that had been used, and the walls were covered in scribbles, ranging from ideas to mere jokes - as serious as this was, keeping team spirit up is crucial in any team activity, and when it can be combined with productivity, creative outlet is a powerful weapon. A man with a weapon may be dangerous, but a man with a weapon and enjoying it is downright lethal. 

On one of the walls, a scribbled Gantt-chart covered in lines and arrows could be seen fading away; nevertheless, this chart had been followed meticulously by the team, and while half the team had progressed from passive recon into active recon, the other half had begun making use of the information to begin weaponizing it. 

So far, there were no signs of detection, and this phase of the process was about to come to an end. It had been easy. In fact, it had been too easy. At first they had spent most of their time going through document after document, creating graphs of which reflections could be seen drawn on the walls, picturing employees and imagining their relationships with one another, trying to find the weakest link. In the end, it turned out their weakest link was their collective pride. 

Bookface had been so proud of their product that before the release of their last big update, they had posted videos of the core team working on the last pieces before release, streaming it on YouTube. As they were watching this, a voice could be heard from the other side of the room: \textit{"Haha, you can almost see what keys he's pressing! I BET THE IDIOT IS TYPING HIS ROOT PASSWORD"}, it said jokingly. After 12 seconds of laughter followed by three seconds of intense eye contact, the competition to figure out what keys were pressed had begun as if planned - there was no reason to take the chance. What if this was the case? It took less than five minutes before the seven people in the Epsilon sub-group had the entire key-sequence figured out.

"s ALT Tab mouseclick s s h whitespace r o o t @ 1 7 4 . 5 8 . 1 3 . 2 1 whitespace - p 5 2 5 2 3 x 1 5 7 1 4 l 1 5 m c d" - there they stopped. They had opened a terminal, and accessed one of the servers with the root user, using the term \text{existentialism} in leetspeak in an attempt to make it look somewhat like hex. \textit{"Aren't you the coolest kid in the cyberworld"}, he thought for himself as he hit the keys and signed on as the privileged root user of whatever Bookface machine he had found.   

As much as this had been a success, the sudden burst of achievement and the feeling of having overpowered the target using only their wits had a blinding effect, and having lost track of time - the dark cellar contributing zero to their timeperception - they had rushed it, and forgot about the timezone difference between their European targets and themselves. It was working hours in Europe, and they had just signed in as root to one of their servers.

Meanwhile in south-eastern Norway, a system administrator was just in the process of updating the kernel to the latest version, addressing serious security flaws, and just as he types in the command to list his working directory, he notices something odd. An oddly named file had appeared in the directory, and his heartbeat increased to a speed matching the hardstyle music of 145 BPM playing in his headphones; quickly he entered \textit{ps aux | grep root}, swiftly followed by lsof piped to an awk command filtering out the PIDs and files being written to. Something was in the process, something was being compiled and he had no control over it. 

If this server was compromised, what servers werent?! He instantly shut the machine down and called his supervisor - he \textit{knew} this was important. An attempt to isolate the threat was made and the entire subnet hosting the machine was immediately disconnected from the other subnets. The next logical step was coordinating database team, storage team and application maintenance team: They must all determine the spread of the threat, what had been compromised and what hadn't, and determine the time of the breach. What if this wasn't as recent as they thought? What if their backups were infected, too? The mood in the building hung like a thick fog in the air, composed of panic and fear inversively proportional to the burst of achievement felt in that basement cellar only moments before in the Alaskan Wilderness. 


\subsection{8. Scenario \#4: Geomagnetic Solarstorm EMP}
This scenario is as an example of an incredibly disastrous event that, however unlikely, has happened in the not so distant past. 

For centuries we have praised the sun as a God in the sky, its power embodied in countless figures and numerous idols across every civilization on this spherical rock spaceship cruising through the ever-expanding universe. We find many names for those we love, as the saying goes - and nothing has ever been as loved and equally feared as the sun. Bringing good harvests, or devastating droughts, this God in the sky has been the focal point of countless cultures, and only in recent times has the extent of this appreciation been lost. 

We know we need it, we know it's there, but its role as the de facto God has gradually diminished - especially since the industrial revolution, and the confidence granted by the rise of technology and only recently, electronics. 

In a world where gluttony is no longer a sin, but the primary way of life in the anglosphere, the devastating effects of the nearly forgotten past God is now overshadowed by the confidence in our species' sustainance through technological means. As with everything, we take it for granted and happily dwell in the goods given by it, thinking that no matter what happens - we'll go through it. We can do it now. We can compensate for any anomaly, and in the minds of the common people is the disbelief in its sudden disappearance. With empirical evidence as the foundation for our new cultural religion of science, scientists have taken the role of past shamans and priests, and the people relying entirely on their mathematical interpretation of the world around them. Surely, now that the world is no longer magical, but proven through the grand Scientific Method, we're no longer blind. But is it really less blinding to take the words of one authority over another, simply because the common belief is that \textit{this} person drew the conclusions from the right evidence?

By basing prophecies on massive amounts of experiments and research, we now refer to them simply as \textit{predictions}, but there is one prediction that is still complicated to make. When will the next devastating solar flare arise, and can we catch it in time? 

A solar flare is not necessarily devastating by default, but a Coronal Mass Ejection is - and the relationship between the two is not fully understood even in this day and age. Maximilian Sentinel was one of the Arch Sciencemen working on this issue, November the 4th, 2019. Only days ago had multiple solar flares erupted on the surface of the sun, to the point where the Committee on Solar and Space Physics wanted to issue an international warning. This was easier said than done though. The government had not been happy about the reports, and were skeptical as to the necessity of such a rare red alert, and the UN remained in disbelief. The BRICS countries being espicially suspicious as the increased tensions since the late 2016 invasion of Iran through Syria led them to believe it was all an elaborate trick to cause panic across the continent and pave the way for a real attack.

So far, nothing had been done about it and nobody knew when, or how big, the next solar flare would be. With the knowledge that most people didn't know, despite rumours spreading rapidly on the internet, Maximilian went to bed that night with a pressure on his chest feeling like that of a truck pressing down upon him. He couldn't sleep. He went down to the kitchen on the bottom floor of the two room apartment on the 39th floor in central Toronto and began making himself a sandwich, when suddenly the lights went off and the power was lost. He felt a rare darkness creep up on him from the corners of his eyes, and as he approached the window overlooking the city, he noticed every single building suffering the same outage. It must be a city-wide blackout he thought, but in the back of his head was another worrying thought - the very reason he couldn't sleep. It wasn't confirmed until numerous airplanes started crashing down from the skies: This was it. Every transistor, in every circuitboard, had been fried by a global electromagnetic pulse and there was no longer any means of warning anyone; it was too late, and every electronic system, everywhere, had been taken out of service - potentially forever.     

He started writing a letter to the database management team of the Committee of Solar and Space Physics, before he realized it was of no use - there were no longer any postoffice, and with the entire world in disarray, there was no way to get any message through but by pushing air from his mouth onto someone else's eardrum; speaking to them in person. Suddenly he remembered his position on the 39th floor and how lucky he had been that nothing had struck him - yet. He rushed down the stairs and jumped on his bike: He \textbf{had} to make it to the office. He knew his colleagues were probably on their way there, if they werent there already. At least he knew his Database Maintenance team, possibly as the only entity in the world right now, had printed everything on paper. Their roles as Database and Application maintenance teams in a technological era was potentially over, but what other effects could be expected from this? A physical database, or a new electronical one, must be reconstructed as fast as possible to be able to predict what would happen next and how to deal with it. In no other situation can the importance of a database and storage maintenance team be so vital, as when they lay the foundation for the survival of the human species. 

\subsection{10. wget -O - -q [url] | grep -i 'src="http' | wc -l}
This essentially just fetches a webpage through wget, passes it onto the grep command via pipe, which filters out all lines containing \textit{src="http} (e.g remote images, videos, flash objects, scripts, stylesheets). The key here is that they're all, or most presumably, remote as most sane webdevelopers would use relative paths otherwise. For example, links to pages within the same domain may not be caught, and a lot of other elements than links may be filtered out. Essentially, it counts how many remote references are made by the downloaded page.

To answer the question: Yes, this is a fairly exact test - depending on what it is you want to test. If the purpose of this command was to find out how many lines contain \textit{src="http}, this is perfect. But be aware that it will overlook lines where singlequotes are used. If you want to filter out all (remote or local) images, it's not a very exact test - consider using proper regex.